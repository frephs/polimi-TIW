\documentclass[a4paper,11pt]{article}
\usepackage[T1]{fontenc}
\usepackage[utf8]{inputenc}
\usepackage[italian]{babel}
\usepackage{lmodern}
\usepackage{amsmath}
\usepackage{amsfonts}
\usepackage{amssymb}
\usepackage{amsthm}
\usepackage{graphicx}
\usepackage{color}
\usepackage{xcolor}
\usepackage{url}
\usepackage{textcomp}
\usepackage{hyperref}
\usepackage{parskip}
\usepackage{array}
\usepackage[table]{xcolor}
\usepackage{adjustbox}

\usepackage[left=3.7cm,right=3.7cm]{geometry}

\usepackage{caption}


\newcommand*\ttvar[1]{\texttt{\expandafter\dottvar\detokenize{#1}\relax}}
\newcommand*\dottvar[1]{\ifx\relax#1\else
  \expandafter\ifx\string.#1\string.\allowbreak\else#1\fi
  \expandafter\dottvar\fi}
\renewcommand{\theparagraph}{\Alph{paragraph}}

\newcommand{\sig}[1]{\texttt{\detokenize{#1}}}
\newcommand{\sigxs}[1]{\scriptsize{\sig{#1}}}

\usepackage{tikz}

\newcommand{\proc}[1]{%
  \tikz[baseline=(char.base)]{%
    \node[ fill=gray!20, rounded corners=2pt, inner sep=3pt] (char) {\texttt{\detokenize{#1}}};%
  }%
}


\usetikzlibrary{automata,positioning}


\newcommand{\red}[1]{%
  \tikz[baseline=(char.base)]{%
    \node[ fill=gray!20, rounded corners=5pt, inner sep=4pt, text=red!90] (char) {\texttt{\detokenize{#1}}};%
  }%
}

\definecolor{darkgreen}{RGB}{0,128,0}

\newcommand{\green}[1]{%
  \tikz[baseline=(char.base)]{%
    \node[ fill=gray!20, rounded corners=5pt, inner sep=4pt, text=darkgreen!90] (char) {\texttt{\detokenize{#1}}};%
  }%
}



\newcommand{\blue}[1]{%
  \tikz[baseline=(char.base)]{%
    \node[ fill=gray!20, rounded corners=5pt, inner sep=4pt, text=blue!90] (char) {\texttt{\detokenize{#1}}};%
  }%
}


\newcommand{\orange}[1]{%
  \tikz[baseline=(char.base)]{%
    \node[ fill=gray!20, rounded corners=5pt, inner sep=4pt, text=orange!90] (char) {\texttt{\detokenize{#1}}};%
  }%
}

\newcommand{\purple}[1]{%
  \tikz[baseline=(char.base)]{%
    \node[ fill=gray!20, rounded corners=5pt, inner sep=4pt, text=purple!90] (char) {\texttt{\detokenize{#1}}};%
  }%
}

\newcommand{\ent}[1]{
    \red{#1}
}

\newcommand{\rel}[1]{
    \green{#1}
}

\newcommand{\attr}[1]{
    \blue{#1}
}

\newcommand{\page}[1]{
    \red{#1}
}

\newcommand{\view}[1]{
    \green{#1}
}

\newcommand{\act}[1]{
    \blue{#1}
}

\newcommand{\ev}[1]{
    \orange{#1}
}





\usepackage{listings}
\usepackage{xcolor}

% Define colors for syntax highlighting
\definecolor{mybackground}{rgb}{0.99,0.99,0.99}
\definecolor{mykeyword}{rgb}{0.0,0.0,0.7}
\definecolor{mycomment}{rgb}{0.0,0.5,0.0}

\lstdefinestyle{VHDL}{
    language=VHDL,
    backgroundcolor=\color{mybackground},
    basicstyle=\ttfamily\scriptsize,
    keywordstyle=\color{mykeyword},
    commentstyle=\color{mycomment},
    morecomment=[s][\color{mycomment}]{--},
    breaklines=true,
    xleftmargin=1em,
    xrightmargin=1em,
    aboveskip=1em,
    belowskip=1em,
    frame=none, % Remove the border
}

\title{Tecnologie informatiche per il web - Analisi della specifica di progetto e documentazione \\ \vspace{0.3cm}\large{ Prof. Piero Fraternali - Anno Accademico 2024/2025}
}
\author{Francesco Genovese (codice persona 10765842 - matricola 983101)}
\date{\today}

\begin{document}

\maketitle
\tableofcontents

\begin{abstract}
 Si analizza la specifica di progetto fornita e i requisiti.
\end{abstract}
\newpage
\section{Database requirements}
Legenda:
\ent{Entita'}
\rel{Relazioni}
\attr{Attributi}
\subsection{Specifica}

Un'applicazione web consente la gestione di aste online.

Gli \ent{utenti} accedono tramite login e possono vendere e acquistare articoli all'asta. Ogni utente ha \attr{username}, \attr{password}, \attr{nome}, \attr{cognome} e \ent{indirizzo} (quest'ultimo è usato per la spedizione degli articoli comperati). Ogni \ent{articolo} ha \attr{codice}, \attr{nome}, \attr{descrizione}, \attr{immagine} e \attr{prezzo}.

La HOME page contiene due link, uno per accedere alla pagina VENDO e uno per accedere alla pagina ACQUISTO. La pagina VENDO mostra una lista delle aste \rel{create dall'utente} e non ancora chiuse, una lista delle aste da lui create e chiuse e due form, uno per creare un nuovo articolo e uno per creare una nuova asta per vendere gli articoli dell'utente.

ll prim form consente di inserire nel database un articolo con tutti i suoi dati, che sono obbligatori. Il secondo form mostra l'elenco degli articoli presenti nel database e disponibili per la vendita e dà la possibilità di selezionarne più di uno per creare un'asta che li comprende. 

Un'\ent{asta} è formata da \rel{uno o piu' articoli} messi in vendita, il prezzo iniziale dell'insieme di articoli, il \attr{rialzo minimo} di ogni offerta (espresso come un numero intero di euro) e \attr{una scadenza} (data e ora, es. 19-04-2021 alle 24:00). Il prezzo iniziale dell'asta è ottenuto come somma del prezzo degli articoli compresi nell'offerta. Lo \rel{stesso articolo non puo' essere incluso in aste diverse}. 

Una volta venduto, un articolo non deve essere più disponibile per l'inserimento in ulteriori aste. La lista delle aste nella pagina VENDO è ordinata per data+ora crescente. 

L'elenco riporta: codice e nome degli articoli compresi nell'asta, offerta massima, tempo mancante (numero di giorni e ore) tra il momento (data ora) del login e la data e ora di chiusura dell'asta. 

Cliccando su un'asta nell'elenco compare una pagina DETTAGLIO ASTA che riporta per un'asta aperta tutti i \rel{dati dell'asta e la lista delle offerte} (nome utente, prezzo offerto, data e ora dell'offerta) ordinata per data+ora decrescente. 

Un bottone CHIUDI permette all'utente di \attr{chiudere l' asta} se è giunta l'ora della scadenza (si ignori il caso di aste scadute ma non chiuse dall'utente e non ci si occupi della chiusura automatica di aste dopo la scadenza).

Se l'asta è chiusa, la pagina riporta tutti i dati dell'asta, il nome dell'aggiudicatario, il prezzo finale e \ent{l' indirizzo (fisso)} di spedizione dell'utente. 

La pagina ACQUISTO contiene una form di ricerca per parola chiave. Quando l'acquirente invia una parola chiave la pagina ACQUISTO è aggiornata e mostra un elenco di aste aperte (la cui scadenza è posteriore alla data e ora dell'invio) per cui la parola chiave compare nel nome o nella descrizione di almeno uno degli articoli dell'asta. La lista è ordinata in modo decrescente in base al tempo (numero di giorni e ore) mancante alla chiusura.

Cliccando su un'asta aperta compare la pagina OFFERTA che mostra i dati degli articoli, l'elenco delle \ent{offerte pervenute} in ordine di \attr{data+ora} decrescente e un campo di input per inserire la propria offerta, che deve essere superiore all'offerta massima corrente di un \attr{importo} pari almeno al rialzo minimo. Dopo l'invio dell'offerta la pagina OFFERTA mostra l'elenco delle offerte aggiornate. La pagina ACQUISTO contiene anche un elenco delle offerte aggiudicate all'utente con i dati degli articoli e il prezzo finale.





















\newpage
\section{
Functional requirements
}
Legenda \page{Pagine} \view{View components} \act{Azioni} \ev{Eventi}

\subsection{Specifica}

\subsubsection{Versione HTML pura}
Un'applicazione web consente la gestione di aste online.

 Gli utenti accedono tramite login e possono vendere e acquistare articoli all'asta. Ogni utente ha username, password, nome, cognome e indirizzo (quest'ultimo è usato per la spedizione degli articoli comperati). Ogni articolo ha codice, nome, descrizione, immagine e prezzo.

La \page{HOME} page contiene due link, quando \ev{cliccati}, uno \act{per accedere} alla pagina \view{VENDO} e uno \act{per accedere} alla pagina \view{ACQUISTO}.

La pagina \page{VENDO} mostra una \view{lista delle aste create} dall'utente e non ancora chiuse, una \view{lista delle aste da lui create e chiuse }e due form, uno per creare un nuovo articolo e uno per creare una nuova asta per vendere gli articoli dell'utente.

1. ll \view{primo form}, \ev{premendo submit}, consente di \act{inserire un articolo} nel database con tutti i suoi dati, che sono obbligatori.

2. Il \view{secondo form}, \ev{premendo submit}, mostra l'elenco degli articoli presenti nel database e disponibili per la vendita e dà la possibilità di selezionarne piu' di uno per \act{creare un'asta} che li comprende. 

Un'asta è formata da uno o più articoli messi in vendita, il prezzo iniziale dell'insieme di articoli, il rialzo minimo di ogni offerta (espresso come un numero intero di euro) e una scadenza (data e ora, es. 19-04-2021 alle 24:00). Il prezzo iniziale dell'asta è ottenuto come somma del prezzo degli articoli compresi nell'offerta. Lo stesso articolo non può essere incluso in aste diverse. 

Una volta venduto, un articolo non deve essere più disponibile per l'inserimento in ulteriori aste. La lista delle aste nella pagina VENDO è ordinata per data+ora crescente. 

L'elenco riporta: codice e nome degli articoli compresi nell'asta, offerta massima, tempo mancante (numero di giorni e ore) tra il momento (data ora) del login e la data e ora di chiusura dell'asta. 

Cliccando su un'asta nell'elenco compare una pagina \page{DETTAGLIO ASTA} che riporta per un'asta aperta tutti i dati dell'asta e la lista delle offerte (nome utente, prezzo offerto, data e ora dell'offerta) ordinata per data+ora decrescente. 

\ev{Premere il bottone} \view{CHIUDI} permette all'utente di \act{chiudere l'asta} se è giunta l'ora della scadenza (si ignori il caso di aste scadute ma non chiuse dall'utente e non ci si occupi della chiusura automatica di aste dopo la scadenza).

Se l'asta è chiusa, la pagina riporta tutti i dati dell'asta, il nome dell'aggiudicatario, il prezzo finale e l'indirizzo (fisso) di spedizione dell'utente. 

La pagina \page{ACQUISTO} contiene una \view{form di ricerca per parola chiave}. Quando l'acquirente \ev{invia una parola chiave} \act{la pagina ACQUISTO e' aggiornata} e mostra un \view{elenco di aste aperte} (la cui scadenza è posteriore alla data e ora dell'invio) per cui la parola chiave compare nel nome o nella descrizione di almeno uno degli articoli dell'asta. La lista è ordinata in modo decrescente in base al tempo (numero di giorni e ore) mancante alla chiusura.

\ev{Cliccando su un'asta aperta} compare la pagina \page{OFFERTA} che mostra i \view{dettagli articoli}, l'\view{elenco delle offerte} pervenute in ordine di data+ora decrescente e un \view{campo di input} che \ev{al submit } per \act{inserire l'offerta}, che deve essere superiore all'offerta massima corrente di un importo pari almeno al rialzo minimo. 

Dopo l'\ev{invio dell'offerta} la pagina OFFERTA mostra \view{l'elenco delle offerte aggiornate}. La pagina ACQUISTO contiene anche un \view{elenco delle offerte aggiudicate} all'utente con i dati degli articoli e il prezzo finale.

\subsubsection{Versione con JavaScript}

Si realizzi un'applicazione client server web che estende e/o modifica le specifiche precedenti come segue:
- Dopo il login, l'intera applicazione e' realizzata con \page{un'unica pagina.}

- Se l'\ev{utente accede per la prima volta} l'applicazione mostra il contenuto della pagina ACQUISTO.

- Se l'utente \ev{ha gia' usato l'applicazione}, questa mostra il contenuto della pagina VENDO se l'ultima azione dell'utente è stata la creazione di un'asta; altrimenti mostra il contenuto della pagina ACQUISTO con l'elenco (eventualmente vuoto) delle aste su cui l'utente ha cliccato in precedenza e che sono ancora aperte.

- L'informazione dell'ultima azione compiuta e delle aste visitate è memorizzata a lato client per la durata di un mese.

- Ogni interazione dell'utente è gestita senza ricaricare completamente la pagina, ma produce l'invocazione asincrona del server e l'eventuale modifica solo del contenuto da aggiornare a seguito dell'evento.

\subsection{Functional analysis summary}

\subsubsection{Pages and view components}
Legenda \page{pagine} \view{view component} \blue{javascript version only} \orange{extra}
\begin{itemize}
\item \page{HOME} $\to$ /welcome
    \begin{itemize}
        \item \view{link VENDO} 
        \item \view{link ACQUISTO}
        \item \orange{link ACCOUNT}
    \end{itemize}
\item \page{VENDO} $\to$ /sell
    \begin{itemize}
        \item \view{lista aste create}
        \item \view{lista aste chiuse}
        \item \view{form CARICA articolo}
        \item \view{form CREA asta}
        \item \orange{form MODIFICA dettagli articoli}
    \end{itemize}
\item \page{DETTAGLIO ASTA} $\to$ /sell/auction/
\begin{itemize}
    \item \view{dettagli dell'asta}
    \item \view{lista offerte}
    \item \view{bottone CHIUDI}
    \item \orange{bottone ELIMINA}
    \item \orange{form MODIFICA dettagli articoli}
\end{itemize}
\item \page{ACQUISTO} $\to$ /buy, /buy/search
\begin{itemize}
    \item \view{form RICERCA}
    \item \view{lista aste risultati ricerca}
    \item \view{lista offerte aggiudicate}
    \item \blue{RIA: lista prodotti cercati in precedenza}
\end{itemize}

\item \page{OFFERTA}  $\to$ /buy/auction/
\begin{itemize}
    \item \view{dettagli articoli}
    \item \view{elenco offerte}
    \item \view{form NUOVA offerta}
\end{itemize}
\item \page{ACCOUNT}  $\to$ /account/
\begin{itemize}
    \item \view{form LOGIN}
    \item \orange{form SIGNIN}
\end{itemize}

\item \page{DETTAGLI ACCOUNT}  $\to$ /account/details
\begin{itemize}
    \item \view{bottone LOGOUT}
    \item \orange{form ELIMINA account}
    \item \orange{form ELIMINA aste}
    \item \orange{form ELIMINA prodotti}
    \item \orange{form MODIFICA dettagli utente}
\end{itemize}
\end{itemize}


\newpage
\subsubsection{Events and corresponding actions}
Legenda \page{pagine} \act{azioni} \ev{eventi} \purple{extra}
\begin{itemize}
    \item \ev{Submit form LOGIN} 
    \begin{itemize}
        \item \act{Verifica} credenziali
        \item Se proveniente da un redirect, \act{reindirizzato} alla sezione di origine.
        \item Altrimenti \act{caricato} \page{/ACCOUNT/DETAILS} o il \page{CONTROLLER} nel caso di RIA
    \end{itemize}

    \item \ev{Submit form SIGNUP} 
    \begin{itemize}
        \item \act{Inserimento} dettagli nel database
        \item \act{Ricarica} la pagina \page{/ACCOUNT} con un messaggio di successo
    \end{itemize}

    \item \ev{Submit form LOGOUT} 
    \begin{itemize}
        \item La sessione dell'utente viene \act{invalidata}.
        \item \act{Reindirizzamento} a \page{/WELCOME} con messaggio di successo.
    \end{itemize}

    \item \ev{Submit form ADD product} 
    \begin{itemize}
        \item \act{Inserimento} nel database dopo la validazione.
        \item \page{/SELL} viene \act{ricaricata}.
    \end{itemize}

    \item \ev{Submit form CREATE auction} 
    \begin{itemize}
        \item \act{Creazione} del record per l'asta dopo la validazione
        \item \act{Aggiornamento} dei valori FK di auction dei prodotti nel database
        \item \page{/SELL} viene \act{ricaricata}.
    \end{itemize}

    \item \ev{Submit form CLOSE auction} 
    \begin{itemize}
        \item \act{Aggiornamento} del valore \ttvar{is_closed} per l'asta
        \item \page{/SELL} viene \act{ricaricata}.
    \end{itemize}

    \item \ev{Submit form ADD bid} 
    \begin{itemize}
        \item \act{Creazione} del record per l'offerta
        \item \page{/BUY/AUCTION/} viene \act{ricaricata}.
    \end{itemize}

    \item \ev{Submit form SEARCH auction} 
    \begin{itemize}
        \item \act{Query} al database
        \item \page{/BUY/SEARCH} viene \act{ricaricata}.
    \end{itemize}

    \item \ev{Click Link Auction DETAILS}
    \begin{itemize}
        \item Vengono \act{recuperati} i dettagli dell'asta e le offerte associate. Per il venditore viene caricata \page{/SELL/AUCTION} per un compratore \page{/BUY/AUCTION}
    \end{itemize}

    \item \purple{Submit form UPDATE username}
    \begin{itemize}
        \item \act{Verifica} della password 
        \item Lo username viene \act{aggiornato} nel database.
        \item \page{/ACCOUNT/DETAILS} viene \act{ricaricata} con un messaggio di successo.
    \end{itemize}

    \item \purple{Submit form UPDATE password}
    \begin{itemize}
        \item \act{Verifica} della password corrente 
        \item La password viene \act{aggiornata} nel database.
        \item \page{/ACCOUNT/DETAILS} viene \act{ricaricata} con un messaggio di successo.
    \end{itemize}

    \item \purple{Submit form UPDATE account details}
    \begin{itemize}
        \item I dati personali e l'indirizzo vengono \act{aggiornati} nel database.
        \item \page{/ACCOUNT/DETAILS} viene \act{ricaricata} con un messaggio di successo.
    \end{itemize}

    \item \purple{Submit form DELETE account} 
    \begin{itemize}
        \item \act{Verifica} della password e \act{rimozione} dell'account dal database.
        \item L'utente viene \act{reindirizzato} a \page{/WELCOME} con un messaggio di successo.
    \end{itemize}

    \item \purple{Submit form DELETE auction} 
    \begin{itemize}
        \item L'asta viene \act{eliminata} dal database se non ci sono offerte.
        \item \page{/SELL} viene \act{ricaricata} con un messaggio di successo.
    \end{itemize}

    \item \purple{Submit form DELETE all auctions} 
    \begin{itemize}
        \item Tutte le aste dell'utente vengono \act{eliminate} dal database se non ci sono offerte.
        \item \page{/ACCOUNT/DETAILS} viene \act{ricaricata} con un messaggio di successo.
    \end{itemize}

    \item \purple{Submit form EDIT product} 
    \begin{itemize}
        \item I dettagli del prodotto vengono \act{aggiornati} nel database.
        \item \page{/SELL} viene \act{ricaricata} con un messaggio di successo.
    \end{itemize}

    \item \purple{Submit form DELETE product} 
    \begin{itemize}
        \item Il prodotto viene \act{eliminato} dal database se non appartiene ad un'asta.
        \item \page{/SELL} viene \act{ricaricata} con un messaggio di successo.
    \end{itemize}

    \item \purple{Submit form DELETE all products} 
    \begin{itemize}
        \item Tutti i prodotti dell'utente vengono \act{eliminati} dal database se non appartengono ad un'asta.
        \item \page{/ACCOUNT/DETAILS} viene \act{ricaricata} con un messaggio di successo.
    \end{itemize}
\end{itemize}

\end{document}



